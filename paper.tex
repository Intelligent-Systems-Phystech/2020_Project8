\documentclass[12pt, twoside]{article}
\usepackage{jmlda}
\newcommand{\hdir}{.}

\begin{document}

\title
    [Порождение признаков с помощью локально-аппроксимирующих моделей] % краткое название; не нужно, если полное название влезает в~колонтитул
    {Порождение признаков с помощью локально-аппроксимирующих моделей}
\author
    [М.\,Е.~Христолюбов] % список авторов (не более трех) для колонтитула; не нужен, если основной список влезает в колонтитул
    {М.\,Е.~Христолюбов} % основной список авторов, выводимый в оглавление
    [М.\,Е.~Христолюбов$^1$] % список авторов, выводимый в заголовок; не нужен, если он не отличается от основного
\email
    {khristolyubov.me@phystech.edu}
\thanks
    {Работа выполнена при
     %частичной
     финансовой поддержке РФФИ, проекты \No\ \No 00-00-00000 и 00-00-00001.}
\organization
    {$^1$Московский физико-технический институт}
\abstract
    {В работе рассматривается многоклассовая классификация временных рядов. Классификация производится методам порождения признаков с помощью локально аппроксимирующих моделей. Предполагается, что временные ряды измерений содержат кластеры в пространстве описаний временных рядовю Поэтому целесообразно разбить временные ряды на сегменты, а в качестве признаков использовать параметры моделей, обученных на этих сегментах. Изучается выполнимость гипотезы о простоте выборки для порожденных признаков и информативность порожденных признаков. В работе предлагается (?новинка).
	
\bigskip
\noindent
\textbf{Ключевые слова}: \emph {временной ряд; многоклассовая классификация; сегментация временных рядов; локально аппроксимирующая модель}
}

\titleEng
	[JMLDA paper template] % краткое название; не нужно, если полное название влезает в~колонтитул
    {Machine Learning and Data Analysis journal paper template}
\authorEng
	[F.\,S.~Author] % список авторов (не более трех) для колонтитула; не нужен, если основной список влезает в колонтитул
	{F.\,S.~Author, F.\,S.~Co-Author, and F.\,S.~Name} % основной список авторов, выводимый в оглавление
    [F.\,S.~Author$^1$, F.\,S.~Co-Author$^2$, and F.\,S.~Name$^{1, 2}$] % список авторов, выводимый в заголовок; не нужен, если он не отличается от основного
\thanksEng
    {The research was
     %partially
    	 supported by the Russian Foundation for Basic Research (grants 00-00-0000 and 00-00-00001).
    }
\organizationEng
    {$^1$Organization, address; $^2$Organization, address}
\abstractEng
    {This is the template of the paper submitted to the journal ``Machine Learning and Data Analysis''.
		
	\noindent
	The title should be concise and informative. Titles are often used in information-retrieval systems. Avoid abbreviations and formulae where possible.
	
	\noindent
	A concise and factual abstract is required.
	
	\noindent
	\textbf{Background}: One paragraph about the problem, existent approaches and its limitations.
	
	\noindent
	\textbf{Methods}: One paragraph about proposed method and its novelty.
	
	\noindent
	\textbf{Results}: One paragraph about major properties of the proposed method and experiment results if applicable.
	
	\noindent
	\textbf{Concluding Remarks}: One paragraph about the place of the proposed method among existent approaches.
		
	\noindent
		
	\noindent
    	\textbf{Keywords}: \emph{keyword; keyword; more keywords, separated by ``;''}}

%данные поля заполняются редакцией журнала
\doi{00.00000/00000000}
\receivedRus{00.00.0000}
\receivedEng{January 00, 0000}

\maketitle
\linenumbers

\section{Введение}
В статье изучается задача идентификации движений человека по временным рядам. Данные представляют измерения акселерометра и гироскопа, встроенных в мобильное устройство IPhone 6s, хранящегося в переднем кармане участника. Метками классов служат: подъем по лестнице вверх, спуск по лестнице вниз, ходьба, бег трусцой, сидение, лежание. В дополнении к этому исследуется возможность выделение атрибутивныч паттернов, которые могут быть использованы для определения пола или личности субъектов данных в дополнение к их деятельности.

Базовый метод решения задачи берется из \cite{Ivkin15} и \cite{Karasikov16}. В работе \cite{Ivkin15} апроксимирующая модель обучается на всем временном ряду без сегментации. В качестве признаков используются параметры модели авторегрегрессии, а так же собственные числа траекторной матрицы, в случае модели сингулярного спектра. В работе \cite{Karasikov16} проводится сегментация временных рядов, а так же сравнивается алгоритм голосования сегментов и классификация в пространстве параметров распределений признаков их сегментов. В работе \cite{Dafne19} используются скрытые марковские модели, но особенность этих задач заключается в классификации действий в каждый момент времени, а не в классификация отдельных временных рядов.

Классификация временых рядов осложненна тем, что исходное описание временного ряда (последовательность значений, снятых с определенной частотой) нельзя использовать в качестве признаков, в связи с тем, что отдельные значения несут мало информации о характерном виде временного ряда. Требуется найти адекватное признаковое пространство, удобное для применения моделей классификации. В качестве признаков могут использльзоваться значения некоторых функций, задаваемые экспертом (дисперсия, минимальное и максимальное значения, среднее значения и т.п.). Однако этот качество этого метода зависит от содержательности экспертных функций, и, зачастую, дает посредественные результаты. Кроме того, в качестве признаков могут изпользоваться коэффициенты разложения в ряд Фурье, сингулярный спектр. 

Еще одним подходом является обучение некоторой модели, апроксимирующей временной рял, и отображение временного ряда в пространство параметров обученной модели. В качестве модели, параматры которой беруться в качестве признаков, можно взять линейную модель, модель авторегрессии. Для этого методы требуется сегментация временного ряда, и способ сегментации является важной проблемой. В работе \cite{Karasikov16} применяется сегментация на фрагменты фиксированной длины.

В работе предлагается оптимальный способ сегментации и метод выделеления некоторых элементарных движений, по признаковому описанию которых можно будет идентифицировать род дейятельности человека соответсующий временному ряду.

\section{Название раздела}


\paragraph{Название параграфа}
Разделы и~параграфы, за исключением списков литературы, нумеруются.

\section{Заключение}
Желательно, чтобы этот раздел был, причём он не~должен дословно повторять аннотацию.
Обычно здесь отмечают, каких результатов удалось добиться, какие проблемы остались открытыми.

%%%% если имеется doi цитируемого источника, необходимо его указать, см. пример в \bibitem{article}
%%%% DOI публикации, зарегистрированной в системе Crossref, можно получить по адресу http://www.crossref.org/guestquery/
\begin{thebibliography}{99}

\bibitem{article}
    \BibAuthor{Карасиков~М.Е., Стрижов~В.В.}
    Классификация временных рядов в пространстве параметров порождающих моделей~//
    \BibJournal{Информатика и ее применения}, 2016.
	\BibDoi{10.3114/S187007708007}.
 	
\end{thebibliography}

\maketitleSecondary
\English
\begin{thebibliography}{99}

\bibitem{Ivkin15}
	\BibAuthor{N.~P.~Ivkin, M.~P.~Kuznetsov.}. 2015.
	 Time series classification algorithm using combined feature description. .
	\BibJournal{Machine Learning and Data Analysis} (11):1471–1483.

\bibitem{Karasikov16}
	\BibAuthor{V.~V.~Strijov, M.~E.~Karasikov.} 2016.
	Feature-based time-series classification
	\BibJournal{Informatics}

\bibitem{Isachenko16}
	\BibAuthor{V.V. Strijov, R.V. Isachenko.}. 2016.
	 Metric learning in multiclass time series classification problem.
	\BibJournal{Informatics and Applications} (10(2)):48–57.

\bibitem{Popova16}
	\BibAuthor{V.V. Strijov, Andrew~Zadayanchuk, Maria~Popova.}. 2016.
	 Selection of optimal physical activity classification model using measurements of accelerometer.
	\BibJournal{Information Technologies}  (22(4)):313–318.

\bibitem{Motrenko16}
	\BibAuthor{Strijov~V.V., Motrenko~A.P.}. 2016.
	 Extracting fundamental periods to segment human motion time series.
	\BibJournal{Journal of Biomedical and Health Informatics}  20(6):1466 – 1476.

\bibitem{Ignatov15}
	\BibAuthor{Strijov~V.V., Ignatov A.}. 2015.
	 Human activity recognition using quasiperiodic time series collected from a single triaxial accelerometer.
	\BibJournal{Multimedia Tools and Applications}  pages 1–14.

\bibitem{Dafne19}
	\BibAuthor{Dafne van Kuppevelt, Joe Heywood, Mark Hamer, Séverine Sabia, Emla Fitzsimons, Vincent van Hees}. 2019.
	 Segmenting accelerometer data from daily life with unsupervised machine learning.
	\BibJournal{PLOS ONE} 
    \BibDoi{10.5255/UKDA-SN-8156-3}.

\bibitem{Malekzadeh19}
	\BibAuthor{Malekzadeh, Mohammad and Clegg, Richard G. and Cavallaro, Andrea and Haddadi, Hamed}. 2019.
	\BibTitle{Mobile Sensor Data Anonymization}  pages 49--58.
    \Bibbooktitle{Proceedings of the International Conference on Internet of Things Design and Implementation}
    \BibDoi{10.1145/3302505.3310068}.

\printbibliography
  	     	
\end{thebibliography}

\end{document}
