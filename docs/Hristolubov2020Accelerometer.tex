\documentclass[12pt, twoside]{article}
\usepackage{jmlda}
\newcommand{\hdir}{.}

\usepackage{dsfont}

\begin{document}

\title
    [Порождение признаков с помощью локально-аппроксимирующих моделей] % краткое название; не нужно, если полное название влезает в~колонтитул
    {Порождение признаков с помощью локально-аппроксимирующих моделей}
\author
    [М.\,Е.~Христолюбов] % список авторов (не более трех) для колонтитула; не нужен, если основной список влезает в колонтитул
    {М.\,Е.~Христолюбов} % основной список авторов, выводимый в оглавление
    [М.\,Е.~Христолюбов$^1$] % список авторов, выводимый в заголовок; не нужен, если он не отличается от основного
\email
    {khristolyubov.me@phystech.edu}
\thanks
    {Работа выполнена при
     %частичной
     финансовой поддержке РФФИ, проекты \No\ \No 00-00-00000 и 00-00-00001.}
\organization
    {$^1$Московский физико-технический институт}
\abstract
    {В работе рассматривается многоклассовая классификация временных рядов акселерометра. Классификация производится методам порождения признаков с помощью локально аппроксимирующих моделей. Предполагается, что точки временного ряда можно разбить на кластеры, соответствующие разным классам. Предлагается выделить квазипериодические сегменты из временных интервалов точек, принадлежащих одному кластеру. В качестве признаков для классификации использовать параметры моделей, обученных на этих сегментах. Исследуется вопрос информативности порожденных признаков и возможность идентификации по ним владельца прибора или его физических параметров.
	
\bigskip
\noindent
\textbf{Ключевые слова}: \emph {временной ряд; многоклассовая классификация; кластеризация; сегментация временного ряда; локально аппроксимирующая модель}
}

\titleEng
	[JMLDA paper template] % краткое название; не нужно, если полное название влезает в~колонтитул
    {Machine Learning and Data Analysis journal paper template}
\authorEng
	[F.\,S.~Author] % список авторов (не более трех) для колонтитула; не нужен, если основной список влезает в колонтитул
	{F.\,S.~Author, F.\,S.~Co-Author, and F.\,S.~Name} % основной список авторов, выводимый в оглавление
    [F.\,S.~Author$^1$, F.\,S.~Co-Author$^2$, and F.\,S.~Name$^{1, 2}$] % список авторов, выводимый в заголовок; не нужен, если он не отличается от основного
\thanksEng
    {The research was
     %partially
    	 supported by the Russian Foundation for Basic Research (grants 00-00-0000 and 00-00-00001).
    }
\organizationEng
    {$^1$Organization, address; $^2$Organization, address}
\abstractEng
    {This is the template of the paper submitted to the journal ``Machine Learning and Data Analysis''.	
	\noindent
	The title should be concise and informative. Titles are often used in information-retrieval systems. Avoid abbreviations and formulae where possible.
	\noindent
	A concise and factual abstract is required.
	\noindent
	\textbf{Background}: One paragraph about the problem, existent approaches and its limitations.
	\noindent
	\textbf{Methods}: One paragraph about proposed method and its novelty.
	\noindent
	\textbf{Results}: One paragraph about major properties of the proposed method and experiment results if applicable.
	\noindent
	\textbf{Concluding Remarks}: One paragraph about the place of the proposed method among existent approaches.	
	\noindent	
	\noindent
    	\textbf{Keywords}: \emph{keyword; keyword; more keywords, separated by ``;''}}

%данные поля заполняются редакцией журнала
\doi{00.00000/00000000}
\receivedRus{00.00.0000}
\receivedEng{January 00, 0000}

\maketitle
\linenumbers

\section{Введение}
В статье изучается задача идентификации движений человека по временным рядам. В дополнении к этому исследуется возможность выделение атрибутивных паттернов, которые могут быть использованы для определения личности или физических параметров субъектов данных в дополнении к их деятельности. Классификация временных рядов находит широкое применение в сфере здравоохранения.

Временные ряды являются объектами сложной структуры, требующие предварительной обработки и представления их в удобном для классификации виде. Необходимо отобразить исходный временной ряд в некоторое пространство признаков. Например, в статье \cite{Ivkin15} временной ряд аппроксимируется моделью, а признаками являются ее параметры. В качестве аппроксимирующей модели берется модель авторегрегрессии, а так же собственные числа траекторной матрицы, в случае модели сингулярного спектра. В работе \cite{Karasikov16} проводится разбиение временных рядов на сегменты фиксированной длины, на которых впоследствии обучается локально-аппроксимирующая модель. Для аппроксимации используется линейная модель, модель авторегрессии и коэффициенты преобразования Фурье. В \cite{Anikeev18} предлагается более разумный способ сегментации, а так же применяется аппроксимации сплайнами. Еще более общий подход к способу сегментации посредством нахождения главных компонент траекторной матрицы, рассмотрен в \cite{Motrenko16}. В \cite{Bochkarev18} сравниваются между собой перечисленные выше подходы.

Однако, вышеперечисленные подходы работают только в случае, когда заранее дан временной ряд, соответствующий одному виду деятельности, что невозможно в реальных условиях. Реальные данные представлены временным рядом, для которого в каждый момент времени нужно определить род деятельности. Метод кластеризации точек, соответствующих участкам разной деятельности, с помощью методы главных компонента (SSA, алгоритм гусеница \cite{Danilov97}) рассмотрен в \cite{Grabovoy20}. На участках, содержащих точки одного кластера, уже можно применять описанные выше методы. Другим подходом к классификации точек временного ряда на основе нейросетей рассмотрены в \cite{Dafne19} и \cite{Cinar18}.

В работе исследуется оптимальный способ кластеризации и сегментации, способ выделения некоторых элементарных движений, по признаковому описанию которых можно будет идентифицировать род деятельности человека. Предлагается построить набор локально-аппроксимирующих моделей и выбрать наиболее адекватные. Производится построение метрического пространство описаний элементарных движений. Новизна работы заключается в исследовании независимости реализаций временного ряда на различных сегментах. Предположительно, выборка не является полностью независимой, а некоторая зависимость между сегментами характеризует физические параметры испытуемого и может быть использована для идентификации. 

\section{Постановка задачи}

Пусть имеется исходный временной ряд $d=\{d_i\}_{i=1}^M\in \mathds{R}^M$. Предполагается, что он состоит из последовательности сегментов: $$d=[s_1,\ldots s_N],$$ где $s_i\in \mathcal{S}$, $|S|$~---~число различных действий (кластеров). Считается, что периоды $|s|$ сегментов различаются незначительно, причем известен максимальный период $|s|\leq T$. Тип активности не меняется часто, то есть можно выделить участки временного ряда, соответствующие одному типу активности.

Требуется решить задачу классификации точек ряда: $$R:\mathcal{I}\rightarrow Y,$$ где $\mathcal{I}=\{1,\ldots M\}$~---~индексы точек ряда, а $Y$~---~метки классов.

Предварительно требуется решить задачу кластеризации, то есть нахождения отображения: $$a:\mathcal{I}\rightarrow Z,$$ где $Z$~---~множество меток кластеров.

Сопоставление меток кластеров и меток классов проведем посредством классификации временных интервалов $x=\{d_t,\ldots d_{t+T}\}$, в которых последовательный набор точек исходного ряда принадлежит одному кластеру: $\forall i\in [t,t+t]:a(d_i)=z_x$. Процедуру выделения из исходного ряда интервалов $x$ посредством кластеризации обозначим $g(d)=\{x_1,\ldots x_P\}$

Пусть $x\in \mathcal{X}$~---~объекты сложной структуры, представленные временными рядами. Рассматривается задача классификации, а именно восстановление зависимости $$y=f(x),$$ где $y\in Y$~---~пространство ответов. Тогда исходная задача классификации представляет собой $R=f\circ g$, где $f$ применяется ко всем интервалам исходного ряда $x\subset d$.

Заданы выборка $\mathfrak{D}$ объектов сложной структуры и ответов $\mathfrak{D}=\{(x_i,y_i)\}_{i=1}^m$. Задача состоит в нахождении функции $f$, минимизирующие суммарные потери на выборке $\mathfrak{D}$, при заданной функция потерь $\mathscr{L}:(\mathcal{X},F,Y)\rightarrow R$, $\mathscr{L}(f(x_i),y_i)$, характеризующая ошибку классификации функции $f\in F$ на элементе $x_i$. 

Пусть $$S:X\rightarrow \mathcal{S}, S(x)=\{s_j(x)\}_{j=1}^{N(x)}$$~---~процедура сегментации, где $s_j(x)$~---~сегменты, возможно, различной длины, и $s_1(x)+\ldots+s_{N(x)}(x)=x$, где $+$ означает конкатенацию.

Пусть $$h:S\rightarrow W = \mathds{R}^n, h(S(x))=w(x)$$~---~процедура построения признакового описания по набору сегментов. Тогда $W$~---~пространство признаков, в котором производится классификация временных рядов.

Пусть $b$~---~алгоритм многоклассовой классификации: $$b:W\rightarrow Y$$

Тогда $f$ ищется в множестве $F$ композиций вида $$f=b\circ h\circ S$$

Функционалом качества является $$Q(f,\mathfrak{D})=\frac{1}{|\mathfrak{D}|}\sum\limits_{(x,y)\in\mathfrak{D}}\mathscr{L}(f(x),y)$$

Для каждой пары $(h,S)$ можно найти оптимальное значение вектора $\hat\mu$ параметров классификатора $b(w(x),\mu)$, минимизирующего функционал качества: $$\hat\mu=\argmin_\mu Q(b\circ h\circ S,\mathfrak{D})$$

Оптимальный метод обучения, задающийся алгоритмом сегментации $S$ и способом задания пространства признаков $h$, определяется по скользящему контролю

$$f_{h,S}^*=\argmin\limits_{h,s}\widehat{CV}(f_{h,S},\mathfrak{D}),$$ где $\widehat{CV}(f,\mathfrak{D})$~---~внешний контроль качества методы обучения $f$, $\mathfrak{D}=\mathfrak{L}\sqcup\mathfrak{E}$:
 
$$\widehat{CV}(\mu, \mathfrak{D})=\frac{1}{r}\sum\limits_{k=1}^r Q(f^*(\mathfrak{L}),\mathfrak{E})$$

В качестве функционала качества используется $$Q(f,\mathfrak{L})=\frac{1}{|\mathfrak{L}|}\sum\limits_{(x,y)\in\mathfrak{L}}|\{(x,y)\in\mathfrak{L}|f(x)=y\}|$$, а в оценке точности классификации объектов класса $c\in Y$ используется модифицированный функционал качества $$Q_c(f,\mathfrak{L})=\frac{1}{|\mathfrak{L}|}\sum\limits_{(x,y)\in\mathfrak{L}}\frac{|\{(x,y)\in\mathfrak{L}|f(x)=y=c\}|}{|\{(x,y)\in\mathfrak{L}|y=c\}|}$$

\section{Базовый вычислительный эксперимент}

Данные для эксперимента представляют собой измерения акселерометра и гироскопа, встроенных в мобильное устройство IPhone 6s, хранящегося в переднем кармане брюк участника. Временные ряды содержат значения ускорения человека и углы ориентацию телефона для каждой из $3$ осей~---~всего $6$ временных рядов. Частота дискретизации составляет $50$ Гц. Метками классов служат: подъем по лестнице вверх, спуск по лестнице вниз, ходьба, бег трусцой, сидение, лежание. Данные собраны с $24$ участников, для каждого из  которых известны рост, вес, возраст и пол. Данные собирались в условиях проведения эксперимента: участникам выдавали телефон и просили выполнять одно из $6$ действий в течении $1-2$ минут.

Цель: 1) проверить адекватность метода кластеризации и сегментации, описанный в \cite{Grabovoy20}, 2) убедится, что модель классификации, обученная на данных одного человека, корректно классифицирует действия другого человека.

Для базового эксперимента берется два одномерных временных ряда в $2000$ точек ($40$ секунд), соответсвтующий одному из углов ориентации телефона 'attitude.roll', на каждом из которых конкретный человек (свой для каждого ряда) сначала выполняет бег, а потом прыжки.

Для кластеризации точек временного ряда используется AgglomerativeClustering, использующий метрику, описанную в постановке задачи, кол-во кластеров $2$ и критерий для слияния кластеров (linkage criteria for the merge strategy) "complete". Для сегментации кластеров используется алгоритм из \cite{Motrenko16}.

На извлеченных сегментах строятся локально-апроксимирующие модели, чьи параметры используются в качестве признаков. Сравниваются информативность признаков, порожденных моделью авторегресси (с $20$ членами), коэффициенты ряда Фурье ($10$ наибольших), $5$ сингулярных чисел SSA разложения. 

Графики: вид кластеризации, сегменты на временном ряду,



\paragraph{Название параграфа}
Разделы и~параграфы, за исключением списков литературы, нумеруются.

\section{Заключение}
Желательно, чтобы этот раздел был, причём он не~должен дословно повторять аннотацию.
Обычно здесь отмечают, каких результатов удалось добиться, какие проблемы остались открытыми.

%%%% если имеется doi цитируемого источника, необходимо его указать, см. пример в \bibitem{article}
%%%% DOI публикации, зарегистрированной в системе Crossref, можно получить по адресу http://www.crossref.org/guestquery/
\begin{thebibliography}{99}
	
 	
\end{thebibliography}

\maketitleSecondary
\English
\begin{thebibliography}{99}

\bibitem{Ivkin15}
	\BibAuthor{N.~P.~Ivkin, M.~P.~Kuznetsov.}. 2015.
	 Time series classification algorithm using combined feature description. .
	\BibJournal{Machine Learning and Data Analysis} (11):1471–1483.

\bibitem{Karasikov16}
	\BibAuthor{V.~V.~Strijov, M.~E.~Karasikov.} 2016.
	Feature-based time-series classification
	\BibJournal{Informatics}
	\BibDoi{10.3114/S187007708007}.
	
\bibitem{Anikeev18}	
	\BibAuthor{D.A. Anikeev, G.O. Penkin, V.V. Strijov}. 2018.
	Local approximation models for human physical activity classification~//
	\BibJournal{Informatics}
	\BibDoi{10.14357/19922264190106}.

\bibitem{Isachenko16}
	\BibAuthor{V.V. Strijov, R.V. Isachenko.}. 2016.
	 Metric learning in multiclass time series classification problem.
	\BibJournal{Informatics and Applications} (10(2)):48–57.

\bibitem{Popova16}
	\BibAuthor{V.V. Strijov, Andrew~Zadayanchuk, Maria~Popova.}. 2016.
	 Selection of optimal physical activity classification model using measurements of accelerometer.
	\BibJournal{Information Technologies}  (22(4)):313–318.

\bibitem{Motrenko16}
	\BibAuthor{Strijov~V.V., Motrenko~A.P.}. 2016.
	 Extracting fundamental periods to segment human motion time series.
	\BibJournal{Journal of Biomedical and Health Informatics}  20(6):1466 – 1476.

\bibitem{Ignatov15}
	\BibAuthor{Strijov~V.V., Ignatov A.}. 2015.
	 Human activity recognition using quasiperiodic time series collected from a single triaxial accelerometer.
	\BibJournal{Multimedia Tools and Applications}  pages 1–14.
	
\bibitem{Bochkarev18}
	\BibAuthor{Isachenko R.V., Bochkarev А.М., Zharikov I.N., Strijov V.V.}. 2018.
	Feature Generation for Physical Activity Classification.
	\BibJournal{Artificial Intelligence and Decision Making}  3 : 20-27.

\bibitem{Dafne19}
	\BibAuthor{Dafne van Kuppevelt, Joe Heywood, Mark Hamer, Séverine Sabia, Emla Fitzsimons, Vincent van Hees}. 2019.
	 Segmenting accelerometer data from daily life with unsupervised machine learning.
	\BibJournal{PLOS ONE}
    \BibDoi{10.5255/UKDA-SN-8156-3}.
    
\bibitem{Sabatini10}
    \BibAuthor{Andrea Mannini, Angelo Maria Sabatini}. 2010.
    Machine Learning Methods for Classifying Human Physical Activity from On-Body Accelerometers
    \BibJournal{PubMed}
    \BibDoi{10.3390/s100201154}.
    
\bibitem{Grabovoy20}
	\BibAuthor{Grabovoy A.V., Strijov V.V}. 2020.
	Quasiperiodic time series clustering for human activity recognition
	\BibJournal{Lobachevskii Journal of Mathematics}
	
\bibitem{Danilov97}
	\BibAuthor{D.L. Danilov and A.A. Zhiglovsky}. 1997.
	\BibTitle{Main components of time series: method "Gesenitsa" (St. Petersburg)}
	
\bibitem{Cinar18}
	\BibAuthor{ Y.G. Cinar and H. Mirisaee}. 2018.
	Period-aware content attention RNNs for time series forecasting with
	missing values
	\BibJournal{”Neurocomputing}  312, 177–186
    
\bibitem{Malekzadeh19}
	\BibAuthor{Malekzadeh, Mohammad and Clegg, Richard G. and Cavallaro, Andrea and Haddadi, Hamed}. 2019.
	\BibTitle{Mobile Sensor Data Anonymization}  pages 49--58.
    \Bibbooktitle{Proceedings of the International Conference on Internet of Things Design and Implementation}
    \BibDoi{10.1145/3302505.3310068}.
 

\printbibliography
  	     	
\end{thebibliography}

\end{document}
